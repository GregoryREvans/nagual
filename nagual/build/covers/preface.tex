\documentclass[11pt]{article}
\usepackage{fontspec}
\usepackage[utf8]{inputenc}
\setmainfont{Bell MT Std}
\usepackage[paperwidth=9in,paperheight=12in,margin=1in,headheight=0.0in,footskip=0.5in,includehead,includefoot,portrait]{geometry}
\usepackage[absolute]{textpos}
\TPGrid[0.5in, 0.25in]{23}{24}
\parindent=0pt
\parskip=12pt
\usepackage{nopageno}
\usepackage{graphicx}
\graphicspath{ {./images/} }
\usepackage{amsmath}
\usepackage{tikz}
\newcommand*\circled[1]{\tikz[baseline=(char.base)]{
            \node[shape=circle,draw,inner sep=1pt] (char) {#1};}}

\begin{document}

\begin{textblock}{23}(0, 1)
\begin{center}
\huge FOREWORD
\end{center}
\end{textblock}

\vspace*{0.25\baselineskip}

\begingroup
\begin{center}
\leftskip0.5in
The \textit{Nagual} is a shapeshifting magician, commonly taking the form of a Jaguar. While some native North American cultures have mythology surrounding the transmutation from human to animal, evidence suggests that this Mesoamerican analog to the European werewolf was, in fact, a cultural import, although this is contested. The Nagual may exhibit traits of either good or evil and is sometimes used as a generic term for ``wizard.'' Carlos Castaneda, whose work is regarded as primarily fictional, defines the Nagual as ``the teacher who becomes the gateway, the doorway, the intermediate between the world of the `seeker' or apprentice, and the world of the spirit.''
\rightskip\leftskip
\phantom{text} \hfill (GRE)
\end{center}
\endgroup


%\vspace*{2\baselineskip}

\begin{center}
\huge INSTRUMENTATION
\end{center}

Flute:
\\
\hspace*{1cm} Alto, Bass
\\
\\
Guitar
\\
\\
Percussion:
\\
\hspace*{1cm} Instruments:
\\
\hspace*{2cm} Bass Drum
\\
\hspace*{2cm} Brake Drum
\\
\hspace*{2cm} ``Gongs'': Large Suspended Cymbal, Small Tam Tam, Medium Tam Tam
\\
\hspace*{2cm} Sandpaper Blocks
\\
\hspace*{2cm} Tom-toms [x3]
\\
\hspace*{2cm} Medium Suspended Cymbal
\\
\hspace*{2cm} Vibraphone
\\
\hspace*{2cm} Wood Blocks [x4] (with thin towel for dampening)
\\
\hspace*{1cm} Implements:
\\
\hspace*{2cm} Bass Drum Mallet
\\
\hspace*{2cm} Bow
\\
\hspace*{2cm} Superball Mallet
\\
\hspace*{2cm} Tam Tam Mallet
\\
\hspace*{2cm} Wire Brushes
\\
\hspace*{2cm} Yarn Mallets
\\
\\
Violin:
\\
\hspace*{1cm} Violin, Viola

%\vspace*{2\baselineskip}

\begin{center}
\huge PERFORMANCE NOTES
\end{center}
\begingroup
\begin{center}

\leftskip0.25in
\pmb{Alternate Timbres} : Rhythmicized timbre alterations are notated as a circled number above a note (such as \circled{1}, \circled{2}, or \circled{3}), where higher numbers refer to a greater deviation in timbre and pitch.
\rightskip\leftskip
\phantom{text} \hfill \phantom{()}

\leftskip0.25in
\pmb{Accidentals} : Accidentals apply only to the pitch which they immediately precede, but persist through ties.
\rightskip\leftskip
\phantom{text} \hfill \phantom{()}
\end{center}
\endgroup

\vspace*{10\baselineskip}

\begin{center}
\leftskip0.25in
\textit{Nagual} was composed for Ensemble Dal Niente as part of the 2021 Summer Residency for New Music at DePaul University.
\rightskip\leftskip
\phantom{text} \hfill \phantom{()}
\end{center}

\vspace*{26\baselineskip}

\begin{center}
duration: c. 7'
\end{center}

\end{document}
