\documentclass[11pt]{article}
\usepackage{fontspec}
\usepackage[utf8]{inputenc}
\setmainfont{Bell MT}
\usepackage[paperwidth=9in,paperheight=12in,margin=1in,headheight=0.0in,footskip=0.5in,includehead,includefoot,portrait]{geometry}
\usepackage[absolute]{textpos}
\TPGrid[0.5in, 0.25in]{23}{24}
\parindent=0pt
\parskip=12pt
\usepackage{nopageno}
\usepackage{graphicx}
\graphicspath{ {./images/} }
\usepackage{amsmath}
\usepackage{tikz}
\newcommand*\circled[1]{\tikz[baseline=(char.base)]{
            \node[shape=circle,draw,inner sep=1pt] (char) {#1};}}

\begin{document}

\begin{textblock}{23}(0, 1)
\begin{center}
\huge FOREWORD
\end{center}
\end{textblock}

\vspace*{0.25\baselineskip}

\begingroup
\begin{center}
\leftskip0.5in
The \textit{Nagual} is a shapeshifting magician, commonly taking the form of a Jaguar. While some native North American cultures have mythology surrounding the transmutation from human to animal, evidence suggests that this Mesoamerican analog to the European werewolf was, in fact, a cultural import, although this is contested. The Nagual may exhibit traits of either good or evil and is sometimes used as a generic term for ``wizard.'' Carlos Castaneda, whose work is regarded as primarily fictional, defines the Nagual as ``the teacher who becomes the gateway, the doorway, the intermediate between the world of the `seeker' or apprentice, and the world of the spirit.''
\rightskip\leftskip
\phantom{text} \hfill (xxxx)
\end{center}
\endgroup


%\vspace*{2\baselineskip}

\begin{center}
\huge INSTRUMENTATION
\end{center}

Flute:
\\
\hspace*{1cm} Alto, Bass
\\
\\
Guitar:
\\
\hspace*{1cm} Acoustic or Electric
\\
\\
Percussion:
\\
\hspace*{1cm} Instruments:
\\
\hspace*{2cm} Bass Drum
\\
\hspace*{2cm} Brake Drum (to be scraped with stone)
\\
\hspace*{2cm} ``Gongs'': Large Suspended Cymbal, Small Tam Tam, Medium Tam Tam
\\
\hspace*{2cm} Sandpaper Blocks
\\
\hspace*{2cm} Tom-toms [x3]
\\
\hspace*{2cm} Medium Suspended Cymbal
\\
\hspace*{2cm} Vibraphone
\\
\hspace*{2cm} Wood Blocks [x4] (with thin towel for dampening)
\\
\hspace*{1cm} Implements:
\\
\hspace*{2cm} Bass Drum Mallet
\\
\hspace*{2cm} Bow
\\
\hspace*{2cm} Stone (scraping implement for brake drum)
\\
\hspace*{2cm} Superball Mallet
\\
\hspace*{2cm} Tam Tam Mallet
\\
\hspace*{2cm} Wire Brushes
\\
\hspace*{2cm} Yarn Mallets
\\
\\
Violin:
\\
\hspace*{1cm} Violin, Viola

%\vspace*{2\baselineskip}

\begin{center}
\huge PERFORMANCE NOTES
\end{center}
\begingroup
\begin{center}

\leftskip0.25in
\pmb{Tempi} : Nearly all tempi in the score are related to one another by metric modulation. It is not the case that every modulation is prepared by the rhythmic key written in the modulation symbol. Rarely are continuous, equidistant beats presented across the modulation boundaries. When the resultant tempo of a modulation is prepared in the preceding section, it is highlighted by the use of a \textit{hauptstimme} bracket. The intention of this notation is not to raise the dynamic level of these passages, but merely to draw attention to their location. Accelerandi and ritardandi are notated by arrows spanning between the starting and ending metronome marks.
\rightskip\leftskip
\phantom{text} \hfill \phantom{()}

\pmb{Repeats} : Two unusual repeats are given in the score: one overlapping repeat and one nested repeat. The units of these complex repeats are distinguished by the color of the repeat-bar symbol.
\rightskip\leftskip
\phantom{text} \hfill \phantom{()}

\pmb{Guitar} : The choice of either acoustic or electric guitar (clean, without effects) may significantly alter the sounding quality of the piece. The \pmb{spazzolato} technique refers to a ``sweeping'' motion of the fingers (or fingernails or plectrum) along the length of the string and \pmb{tremolo} notation on a chord refers to rasgueado. The characteristic tension and material of nylon strings on the acoustic guitar provide the unique character of the rasgueado technique, however spazzolato performed on an acoustic guitar may not have sufficient loudness to be heard above the ensemble. This can be solved either by amplification of the acoustic guitar or by the use of of an electric guitar, whose strings may produce a more textured sound.
\rightskip\leftskip
\phantom{text} \hfill \phantom{()}

\pmb{Percussion} : The \pmb{brake drum} is always to be performed by being scraped by a stone. The motion should be one full circular rotation per notated rhythm.
\rightskip\leftskip
\phantom{text} \hfill \phantom{()}

\pmb{Violin} : \circled{1} The damp sign means three fingers placed harmonic-lightly on the string. The sound is `white' and damped, but with a pitch center to the band of white noise still very much discernible. \circled{2} The diamond symbols refer to two (or more depending on the number of diamonds) fingers placed harmonic-lightly on the string. \circled{3} The parenthetical-diamond spanners are trills between the single-, double- and triple-harmonic-damping techniques. That is: a double-diamond spanner with the top of the two diamonds parenthesized means to place two harmonic-light fingers on the string and then repeatedly lift-and-replace the finger closest to the nut on and off the string; this will effectively trill between vanilla-harmonic and the complex sound the double-harmonic will produce. Likewise, the triple-diamond spanner with the bottom two of the three diamonds parenthesized means to place three harmonic-light fingers on the string and then repeatedly lift-and-replace the bridgemost fingers on and off the string; this will effectively trill between the harmonic and the white-damped sound. \circled{4} The chopped bowing technique is essentially a very short spazzolato motion, traditionally near the frog.
\rightskip\leftskip
\phantom{text} \hfill \phantom{()}

\leftskip0.25in
\pmb{Accidentals} : After temporary accidentals, cancellation marks are printed also in the following measure (for notes in the same octave) and, in the same measure, for notes in other octaves, but they are printed again if the same note appears later in the same measure, except if the note is immediately repeated.
\rightskip\leftskip
\phantom{text} \hfill \phantom{()}
\end{center}
\endgroup

\vspace*{9\baselineskip}

\begin{center}
\leftskip0.25in
\textit{Nagual} was composed for Ensemble Dal Niente as part of the 2021 Summer Residency for New Music at DePaul University.
\rightskip\leftskip
\phantom{text} \hfill \phantom{()}
\end{center}

\vspace*{26\baselineskip}

\begin{center}
duration: c. 7'
\end{center}

\end{document}
